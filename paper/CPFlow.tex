\documentclass[amsfonts, amssymb, aps, nofootinbib]{revtex4-2}
\usepackage[T1]{fontenc}
\usepackage{tgtermes}
\usepackage{amsmath}
\usepackage{empheq}
\usepackage{graphicx}
\usepackage[braket, qm]{qcircuit}
\usepackage{braket}
\usepackage{hyperref}
\hypersetup{
	colorlinks   = true, %Colours links instead of ugly boxes
	urlcolor     = blue, %Colour for external hyperlinks
	linkcolor    = blue, %Colour of internal links
	citecolor   = red %Colour of citations
}


\begin{document}
\title{Optimal variational synthesis of small-scale quantum circuits.}
\begin{abstract}
	We consider the problem of variational unitary synthesis. 
\end{abstract}
\maketitle	
\tableofcontents
\section{Introduction}
\begin{itemize}
	\item Compilation -- translate from high-level algorithm to hardware instructions
	\item Relation to hardware efficient variational algorithms
	
\end{itemize}

\begin{align}
TLB(n) = \frac14\left(4^n-3n-1\right) \ . \label{TLB}
\end{align}

\begin{align}
d(U,V)=1-\frac1{4^n}|\operatorname{Tr}{U^\dagger V}|^2 \ . \label{d hst}
\end{align}

\begin{align}
\scalebox{1.0}{
	\Qcircuit @C=1.0em @R=0.8em @!R {
	 & \ctrl{2} & \qw \\
		CZ=\qquad\qquad& &\qquad\qquad\qquad\qquad\quad	= {\begin{pmatrix}1&0&0&0\\0&1&0&0\\0&0&1&0\\0&0&0&-1\end{pmatrix}}		\\
		& \control\qw & \qw \\
		\\ }} \label{def CZ}
	\\
\scalebox{1.0}{
	\Qcircuit @C=1.0em @R=0.8em @!R { \\
		& \ctrl{2} & \dstick{\hspace{2.0em}\mathrm{P}\,(\mathrm{a})} \qw \\
		CP(a)=\qquad\qquad& &\qquad\qquad\qquad\qquad\qquad\qquad\qquad	= {\begin{pmatrix}1&0&0&0\\0&1&0&0\\0&0&1&0\\0&0&0&e^{i\pi a}\end{pmatrix}}\\
		& \control \qw & \qw\\
		\\ }} \label{def CP}
\end{align}

\subsection{Template circuits}
Following \cite{Madden2021, Rakyta2021} we will construct the template circuits out of two-qubit blocks repeated in a regular manner. The two types of entangling blocks we will use are CZ- and CP-blocks depicted at fig.\ref{fig blocks}. They only differ by the type of the entangling gates used. 
\begin{figure}[h!]
\centering
\scalebox{1.0}{(a)
	\Qcircuit @C=1.0em @R=0.2em @!R { \\
		& \ctrl{1} & \gate{\mathrm{R_X}\,(\mathrm{a_0})} & \gate{\mathrm{R_Z}\,(\mathrm{a_1})} & \qw & \qw\\
		& \control\qw & \gate{\mathrm{R_X}\,(\mathrm{a_2})} & \gate{\mathrm{R_Z}\,(\mathrm{a_3})} & \qw & \qw\\
		\\ }}\qquad\qquad\qquad
\scalebox{1.0}{(b)
	\Qcircuit @C=1.0em @R=0.2em @!R { \\
		& \ctrl{1} & \dstick{\hspace{2.0em}\mathrm{P}\,(\mathrm{a_5})} \qw & \qw & \qw & \gate{\mathrm{R_X}\,(\mathrm{a_0})} & \gate{\mathrm{R_Z}\,(\mathrm{a_1})} & \qw & \qw\\
		& \control \qw & \qw & \qw & \qw & \gate{\mathrm{R_X}\,(\mathrm{a_2})} & \gate{\mathrm{R_Z}\,(\mathrm{a_3})} & \qw & \qw\\
		\\ }}
\caption{(a) CZ block (b) CP block.}
\label{fig blocks}
\end{figure}

The blocks are further arranged in sequences we refer to as \textit{layers}. In principle layers can can be arbitrary, but we will usually identify layers with coupling maps of the target topology. For example fig.\ref{fig layers} shows layers for the fully connected and 1D topology.
\begin{figure}[h!]
\centering
\scalebox{1.0}{(a) Connected layer
	\Qcircuit @C=1.0em @R=0.8em @!R { \\
		 & \ctrl{1} & \ctrl{2} & \ctrl{3} & \qw & \qw & \qw & \qw \\
		& \control\qw & \qw & \qw & \ctrl{1} & \ctrl{2} & \qw & \qw \\
		& \qw & \control\qw & \qw & \control\qw & \qw & \ctrl{1} & \qw\\
		& \qw & \qw & \control\qw & \qw & \control\qw & \control\qw & \qw \\
		\\ }}\qquad\qquad\qquad
\scalebox{1.0}{ (b) Chain layer
	\Qcircuit @C=1.0em @R=0.8em @!R { \\
		& \ctrl{1} & \qw & \qw & \qw\\
		& \control\qw & \ctrl{1} & \qw & \qw\\
		& \qw & \control\qw & \ctrl{1} & \qw\\
		& \qw & \qw & \control\qw & \qw\\
		\\ }}
\caption{(a) 4q connected layer (b) 4q chain layer. Here CZ gates are only meant to specify locations of 2q blocks, not their content. }
\label{fig layers}
\end{figure}

Finally, to fully specify the template one must provide the total \textit{number of 2q gates}. Layers are repeated until the specified number of 2q gates is reached, the last layer is truncated if needed. 
For illustration, fig.\ref{fig template example} depicts the full template circuit corresponding to 3q chain layer of CZ-blocks with 5 2q gates in total.

\begin{figure}[h!]
\scalebox{0.7}{
	\Qcircuit @C=1.0em @R=0.2em @!R { \\
		& \gate{\mathrm{R_Z}\,(\mathrm{a_{0}})} & \gate{\mathrm{R_X}\,(\mathrm{a_{1}})} & \gate{\mathrm{R_Z}\,(\mathrm{a_{2}})} & \ctrl{1} & \gate{\mathrm{R_X}\,(\mathrm{a_{9}})} & \gate{\mathrm{R_Z}\,(\mathrm{a_{10}})} & \qw & \qw & \qw & \ctrl{1} & \gate{\mathrm{R_X}\,(\mathrm{a_{17}})} & \gate{\mathrm{R_Z}\,(\mathrm{a_{18}})} & \qw & \qw & \qw & \ctrl{1} & \gate{\mathrm{R_X}\,(\mathrm{a_{25}})} & \gate{\mathrm{R_Z}\,(\mathrm{a_{26}})} & \qw & \qw\\
		& \gate{\mathrm{R_Z}\,(\mathrm{a_{3}})} & \gate{\mathrm{R_X}\,(\mathrm{a_{4}})} & \gate{\mathrm{R_Z}\,(\mathrm{a_{5}})} & \control\qw & \gate{\mathrm{R_X}\,(\mathrm{a_{11}})} & \gate{\mathrm{R_Z}\,(\mathrm{a_{12}})} & \ctrl{1} & \gate{\mathrm{R_X}\,(\mathrm{a_{13}})} & \gate{\mathrm{R_Z}\,(\mathrm{a_{14}})} & \control\qw & \gate{\mathrm{R_X}\,(\mathrm{a_{19}})} & \gate{\mathrm{R_Z}\,(\mathrm{a_{20}})} & \ctrl{1} & \gate{\mathrm{R_X}\,(\mathrm{a_{21}})} & \gate{\mathrm{R_Z}\,(\mathrm{a_{22}})} & \control\qw & \gate{\mathrm{R_X}\,(\mathrm{a_{27}})} & \gate{\mathrm{R_Z}\,(\mathrm{a_{28}})} & \qw & \qw\\
		& \gate{\mathrm{R_Z}\,(\mathrm{a_{6}})} & \gate{\mathrm{R_X}\,(\mathrm{a_{7}})} & \gate{\mathrm{R_Z}\,(\mathrm{a_{8}})} & \qw & \qw & \qw & \control\qw & \gate{\mathrm{R_X}\,(\mathrm{a_{15}})} & \gate{\mathrm{R_Z}\,(\mathrm{a_{16}})} & \qw & \qw & \qw & \control\qw & \gate{\mathrm{R_X}\,(\mathrm{a_{23}})} & \gate{\mathrm{R_Z}\,(\mathrm{a_{24}})} & \qw & \qw & \qw & \qw & \qw\\
		\\ }}
\caption{Template circuit $U(chain, CZ, 5)$.}
\label{fig template example}
\end{figure}



\section{Challenges to variational synthesis}
\subsection{Generally}
\begin{itemize}
	\item Discrete search over architectures 
	\item Continuous optimization
	\item Overview of existing software: QFAST, QSearch, SQUANDER
\end{itemize}
In variational algorithms: barren plateaus and local minimums. We: only local minimums.
\subsection{Focus on local minimums}



We will quantify the challenges associated with local minimums by the empirical success ratio
\begin{align}
	SR=\frac{M}{N} \ ,
\end{align}
where $N$ is the total number of times the optimization procedure is performed starting with different random initial conditions and $M$ is the number of times a global minimum is reached. 

For example, let $U(a)$ be the unitary matrix of the template circuit from fig.\ref{fig template example} and $a^*$ be some particular choice of angles. It is clear that the global minimum of the Hilbert-Schmidt distance \eqref{d hst} $d(U(a), U(a^*))$ is zero, but gradient-based optimization does not always reach it. For some choice of $a^*$ and out default optimizer specification (\ref) we find that the success ratio in this case is $SR=0.39$ from a 100 samples, which implies that roughly half of the times optimization gets stuck in a local minimum. 

We now extend this simple numerical experiment. Fig.\ref{fig local minumums} charts the success ratios for 3q and 4q circuits as a function of the number of gates. The basic procedure is the same with several additions. First, we used templates with fully connected layers (see fig.\ref{fig layers}) extended to the required number of 2q gates. For each number of 2q gates we take 10 different parameter assignments $a^*$ and compute success ratios for each of them using 2000 initial conditions. Blue markers represent mean success ratios averaged over 10 target circuits, while error bars quantify the standard deviation. Absence of blue markers implies that the the empirical success ratio was zero from 2000 samples. For 4q circuits datapoints were collected only every third integer.

There are several remarkable features that deserve highlighting. First, the success ratio drops very quickly as the 2q gate count increases, reaching values below $10^{-3}$ at 11 CZ gates for 3q circuits and 18 CZ gates fro 4q circuits. Next, perhaps surprisingly, the success ratio rises back and gets close to unity as the number of CZ gates approaches the theoretical lower bound \eqref{TLB} in agreement with the empirical evidence found in the literature \cite{Madden2021, Rakyta2021, Kiani2020}. A plausible explanation for this fact \cite{Ge2022} is that when the template is sufficiently expressive, optimization w.r.t. parameters becomes essentially the same as optimization over the unitary matrices themselves. At the same time, cost functions typically considered in quantum computing are simple, often linear or quadratic functions of the unitary matrix elements and hence are not prone to local minimums. Finally, although there is a certain spread of success ratio across different template instances, dependence on 2q gate count sets the dominating trend.

This suggests that success ratio is mostly a function of the template, not of the target. To confirm this  intuition  we carried out additional experiments using random unitaries $V$ instead of template instances as targets. The difficulty here is that the true value of the global minimum of $d(U(a), V)$ is not known, but the presence of local minimums is still manifest. We modify definition of the success ratio in this case, by counting as successful all optimization runs that approached sufficiently closely the lowest value of $d$ across all runs for a given unitary. Note that with this modified definition the success ratio can never be zero (because there is always at least a single run with the lowest value). We see that in the regime when success ratios for random instances are sufficiently high success ratios for random unitaries closely parallel, both in mean and in deviation. In the regions where success ratios for random instances are zero, success ratios for random unitaries are non-zero (they can not be by construction) but are sufficiently close to zero. We expect them to drop further if more samples are accounted for. Overall, our experiments strongly suggest that local minimums are mostly determined by the templates and not by targets.

Of course, success ratio is not only a function of the loss landscape but also of the optimization algorithm. Mathematically the problem of variational synthesis is very similar to the classical optimization subroutines encountered in variational algorithms \cite{}, especially in the subclass referred to as  hardware-efficient \cite{Kandala2017}\cite{}. The problem of local minimums is well appreciated in the relevant literature. It was shown to be NP-hard in the worst case \cite{Bittel2021}. On the heuristic side there are a number of proposals to alleviate the problem \cite{Wierichs2020, Rivera-Dean2021, ?} but in our experiments none performed sufficiently better than simple ADAM\cite{adam}-based optimization to justify additional computational resources that are typically required.

\begin{figure}
\includegraphics[width=0.4\textwidth]{figures/3q_success_chart}
\includegraphics[width=0.4\textwidth]{figures/4q_success_chart}
\caption{Local minimums as a function of circuit complexity. Datapoints for random unitaries are advanced by a half unit along $x$ axis for clarity.}
\label{fig local minumums}
\end{figure}

\begin{itemize}
	\item 3q and 4q charts.
	\item Separately marked Toffoli gates
\end{itemize}
\section{The CPFlow algorithm}
\subsection{Motivation and overview}
In the context of variational synthesis results of the previous section suggest that solving the continuous optimization problem may be just as difficult as solving the discrete architecture search: even if the structure of the template is a perfect match for the target unitary, finding the suitable angles may be very challenging. In the absence of an efficient way to solve the latter problem in our approach we choose the brute force route of trying many initial conditions. A second main idea of our approach is to reformulate the discrete architecture search as a continuous problem as well \footnote{While this work was in preparation similar idea was independently introduced in \cite{Rakyta2022}.}. For illustration, consider the circuit at fig.\ref{fig cp example}.
\begin{figure}[h!]
\scalebox{0.7}{
	\Qcircuit @C=1.0em @R=0.2em @!R { \\
		 & \gate{\mathrm{R_Z}\,(\mathrm{a_{0}})} & \gate{\mathrm{R_X}\,(\mathrm{a_{1}})} & \gate{\mathrm{R_Z}\,(\mathrm{a_{2}})} & \ctrl{1} & \dstick{\hspace{2.0em}\mathrm{P}\,(\mathrm{a_{13}})} \qw & \qw & \qw & \gate{\mathrm{R_X}\,(\mathrm{a_{9}})} & \gate{\mathrm{R_Z}\,(\mathrm{a_{10}})} & \ctrl{2} & \qw & \qw & \qw & \gate{\mathrm{R_X}\,(\mathrm{a_{14}})} & \gate{\mathrm{R_Z}\,(\mathrm{a_{15}})} & \qw & \qw & \qw & \qw & \qw & \qw & \qw & \qw\\
		& \gate{\mathrm{R_Z}\,(\mathrm{a_{3}})} & \gate{\mathrm{R_X}\,(\mathrm{a_{4}})} & \gate{\mathrm{R_Z}\,(\mathrm{a_{5}})} & \control \qw & \qw & \qw & \qw & \gate{\mathrm{R_X}\,(\mathrm{a_{11}})} & \gate{\mathrm{R_Z}\,(\mathrm{a_{12}})} & \qw & \dstick{\hspace{2.0em}\mathrm{P}\,(\mathrm{a_{18}})} \qw & \qw & \qw & \qw & \qw & \ctrl{1} & \dstick{\hspace{2.0em}\mathrm{P}\,(\mathrm{a_{23}})} \qw & \qw & \qw & \gate{\mathrm{R_X}\,(\mathrm{a_{19}})} & \gate{\mathrm{R_Z}\,(\mathrm{a_{20}})} & \qw & \qw\\
		& \gate{\mathrm{R_Z}\,(\mathrm{a_{6}})} & \gate{\mathrm{R_X}\,(\mathrm{a_{7}})} & \gate{\mathrm{R_Z}\,(\mathrm{a_{8}})} & \qw & \qw & \qw & \qw & \qw & \qw & \control \qw & \qw & \qw & \qw & \gate{\mathrm{R_X}\,(\mathrm{a_{16}})} & \gate{\mathrm{R_Z}\,(\mathrm{a_{17}})} & \control \qw & \qw & \qw & \qw & \gate{\mathrm{R_X}\,(\mathrm{a_{21}})} & \gate{\mathrm{R_Z}\,(\mathrm{a_{22}})} & \qw & \qw\\
		\\ }}
\caption{Exemplary CP template. Draw possible architechres \ref{}}
\label{fig cp example}
\end{figure}

Here 2q gates are controlled phase (CP) gates \eqref{def CP} which interpolate between the identity gate $CP(0)=\mathbb{I}$ and the $CZ$ \eqref{def CZ} gate $CP(\pi)=CZ$. For generic values of the angle the $CP(a)$ gate can be decomposed into 2 $CZ$ gates \ref{}. Therefore, different values of parameters in $CP$ gates in the template \eqref{fig cp example} effectively capture several different templates with $CZ$ 2q gates and training templates with $CP$-blocks can encompass both the architecture search and the tuning of continuous parameters, moreover performed in a coherent manner.


We can anticipate, however, that training $CP$ templates to minimize the Hilbert-Schmidt distance to the target unitary directly will result in most $CP$ gates having generic angles and hence effectively double the $CZ$ count of the original template. To address this issue we introduce additional \textit{penalty term} to the cost function that is intended to drive all $CP$ angles either to $0$ or to $\pi$. The shape of the penalty function that we use is presented at fig.\ref{fig penalty}.
\begin{figure}[h!]
\includegraphics[width=0.4\textwidth]{figures/penalty}
\caption{Penalty function for angles of the $CP$-gates. For clarity of the figure, width of plateaus near $0,\frac12\pi, \frac32\pi, 2\pi$} is twice the value used in numerical experiments.
\label{fig penalty}
\end{figure}

This penalty is intended to drive all $CP$-angles during optimization to either $0$ or $\pi$ and hence to reduce the $CZ$-count of the resulting circuit. For numerical stability small plateaus are added near values $0,\pi$ and also $\frac12\pi, \frac32\pi$  (empirically we find that $CP$ angles are often equilibrate near these values too).

\subsection{Details}
We formulate the general synthesis problem as follows. Let $L(U)$ be the loss function to be minimized with unitary as the argument. For unitary synthesis $L(U)$ may be any measure of fidelity to the target unitary $V$, for example $L(U)=D(U, V)$. For state preparation we can choose $L(U) = \Big|\braket{\psi|U|0}\Big|^2$ where $\ket{\psi}$ is the target state and $\ket{0}$ is the usual reference state, etc. The goal is to find a unitary $U^k_{CZ}(a)$ such that $L(U^k_{CZ}(a))$ is sufficiently close to the global minimum of $L(U)$ and at the same time the number $k$ of 2q gates in $U^k_{CZ}$ is as small as possible. 

The loss function directly addressed by CPFlow is given by
\begin{align}
\mathcal{L}(a)=L(U^k_{CP}(a))+r\sum_{a_i\in CP} R(a_i) \label{CP loss}
\end{align}
The first term is simply the value of the loss function at the given unitary. The second term is the regularization term summing the $CP$-penalties for all $CP$-angles in the circuit. The number of 2q gates $k$ and the overall regularization weight $r$ are two of the most important hyperparameters of the model.

Main stages of the algorithm are as follows.

\begin{enumerate}
\item \textit{Raw sampling.} Loss function \eqref{CP loss} is minimized starting from many initial conditions. Both the $CP$-angles and angles of 1q gates are selected uniformly and independently at random\footnote{Distribution of $CP$ angles could be changed.}. 
\item \textit{Selection.} Results of the first step are selected based on two criteria (i) the original loss function $L(U^k_{CP}(a^*))$ must be below a given threshold (we use $10^-4$) and (ii) the number of gates $\sum_{a_i\in CP}G(a_i)$ is below a specified value. Condition (i) means we only accept circuits that are close enough to the global minimum while (ii) filters out decompositions with too high $CZ$ count.
\item \textit{Verification.} Results selected at the previous step are dubbed as prospective. At this stage they are verified more...
\end{enumerate}




In the previous sections we painted the problem of local minimums as a very real obstacle to finding optimal circuits

\begin{itemize}
	\item Aiming to solve both discrete and continuous optimization at once
	\item Local minimums addressed in a direct way -- trying many initial conditions
	\item Better than random search
	\item Hyperparameters are essential
	\item Bayesian optimization of hyperparameters
	\item Exact synthesis
\end{itemize}
\section{Benchmarks}
\subsection{Clifford+T from database}
\subsection{State prep for error correction}
\section{Synthesis of special gates}
\subsection{Toffoli 3 on chain up to SWAP}
\subsection{Toffoli 4 on all topologies including new record at T-shaped}
\subsection{Toffoli 5 already difficult, can be done separating into parts}

\bibliography{/home/idnm/Dropbox/hep/Sheets/library.bib}
\end{document}

\begin{figure}
	
	\centering{\text{Template circuit}}
	
	\scalebox{0.7}{
		\Qcircuit @C=1.0em @R=0.2em @!R { \\
			& \gate{\mathrm{R_Z}\,(\mathrm{a_{0}})} & \gate{\mathrm{R_X}\,(\mathrm{a_{1}})} & \gate{\mathrm{R_Z}\,(\mathrm{a_{2}})} & \ctrl{1} & \gate{\mathrm{R_X}\,(\mathrm{a_{9}})} & \gate{\mathrm{R_Z}\,(\mathrm{a_{10}})} & \ctrl{2} & \gate{\mathrm{R_X}\,(\mathrm{a_{13}})} & \gate{\mathrm{R_Z}\,(\mathrm{a_{14}})} & \qw & \qw & \qw & \ctrl{1} & \gate{\mathrm{R_X}\,(\mathrm{a_{21}})} & \gate{\mathrm{R_Z}\,(\mathrm{a_{22}})} & \qw & \qw\\
			& \gate{\mathrm{R_Z}\,(\mathrm{a_{3}})} & \gate{\mathrm{R_X}\,(\mathrm{a_{4}})} & \gate{\mathrm{R_Z}\,(\mathrm{a_{5}})} & \control\qw & \gate{\mathrm{R_X}\,(\mathrm{a_{11}})} & \gate{\mathrm{R_Z}\,(\mathrm{a_{12}})} & \qw & \qw & \qw & \ctrl{1} & \gate{\mathrm{R_X}\,(\mathrm{a_{17}})} & \gate{\mathrm{R_Z}\,(\mathrm{a_{18}})} & \control\qw & \gate{\mathrm{R_X}\,(\mathrm{a_{23}})} & \gate{\mathrm{R_Z}\,(\mathrm{a_{24}})} & \qw & \qw\\
			& \gate{\mathrm{R_Z}\,(\mathrm{a_{6}})} & \gate{\mathrm{R_X}\,(\mathrm{a_{7}})} & \gate{\mathrm{R_Z}\,(\mathrm{a_{8}})} & \qw & \qw & \qw & \control\qw & \gate{\mathrm{R_X}\,(\mathrm{a_{15}})} & \gate{\mathrm{R_Z}\,(\mathrm{a_{16}})} & \control\qw & \gate{\mathrm{R_X}\,(\mathrm{a_{19}})} & \gate{\mathrm{R_Z}\,(\mathrm{a_{20}})} & \qw & \qw & \qw & \qw & \qw\\
			\\ }}
	
	\centering{\text{Target circuit}}	
	
	\scalebox{0.7}{
		\Qcircuit @C=1.0em @R=0.2em @!R { \\
			& \gate{\mathrm{R_Z}\,(\mathrm{6.246})} & \gate{\mathrm{R_X}\,(\mathrm{0.5124})} & \gate{\mathrm{R_Z}\,(\mathrm{1.219})} & \ctrl{1} & \gate{\mathrm{R_X}\,(\mathrm{1.903})} & \gate{\mathrm{R_Z}\,(\mathrm{5.638})} & \ctrl{2} & \gate{\mathrm{R_X}\,(\mathrm{2.711})} & \gate{\mathrm{R_Z}\,(\mathrm{5.384})} & \qw & \qw & \qw & \ctrl{1} & \gate{\mathrm{R_X}\,(\mathrm{1.105})} & \gate{\mathrm{R_Z}\,(\mathrm{1.16})} & \qw & \qw\\
			& \gate{\mathrm{R_Z}\,(\mathrm{4.646})} & \gate{\mathrm{R_X}\,(\mathrm{2.216})} & \gate{\mathrm{R_Z}\,(\mathrm{1.714})} & \control\qw & \gate{\mathrm{R_X}\,(\mathrm{2.833})} & \gate{\mathrm{R_Z}\,(\mathrm{0.6454})} & \qw & \qw & \qw & \ctrl{1} & \gate{\mathrm{R_X}\,(\mathrm{2.812})} & \gate{\mathrm{R_Z}\,(\mathrm{2.779})} & \control\qw & \gate{\mathrm{R_X}\,(\mathrm{2.515})} & \gate{\mathrm{R_Z}\,(\mathrm{2.283})} & \qw & \qw\\
			& \gate{\mathrm{R_Z}\,(\mathrm{6.106})} & \gate{\mathrm{R_X}\,(\mathrm{2.695})} & \gate{\mathrm{R_Z}\,(\mathrm{1.408})} & \qw & \qw & \qw & \control\qw & \gate{\mathrm{R_X}\,(\mathrm{5.926})} & \gate{\mathrm{R_Z}\,(\mathrm{0.1336})} & \control\qw & \gate{\mathrm{R_X}\,(\mathrm{3.677})} & \gate{\mathrm{R_Z}\,(\mathrm{1.53})} & \qw & \qw & \qw & \qw & \qw\\
			\\ }}
	
	\caption{Exemplary template and target circuits.}	
	\label{fig SR circuits}
\end{figure}